\documentclass[12pt,a4paper]{article}
\usepackage{polyglossia}
\setdefaultlanguage{english}
\usepackage[margin=1cm]{geometry}
\usepackage{amsmath,amsthm,amssymb}
\usepackage{bussproofs}

\newcommand{\N}{\mathbb{N}}
\newcommand{\B}{\mathbb{B}}

\title{Concurrent-\(\lambda\) Operational Semantics}
\author{Antonin Décimo}
\date{}

\begin{document}
\maketitle
\thispagestyle{empty}

To describe the operational semantics of the concurrent-lambda
calculus, we use tuples of the form \((e, s, h, \sigma)\) where \(e\)
is an expression (an extended lambda-term), \(s\) an environment
associating variables to values, \(h\) a growable heap associating
addresses to values, and \(\sigma\) is a list of threads, that are
represented as pairs of extended lambda-terms and their environment.
Extended lambda-terms evaluate to tuples of value, heap, and thread
list.

Note that binary operators are not standard lambda-calculus functions,
thus it not possible to do ML-style partial applications with them.

\begin{center}
  \AxiomC{}
  \LeftLabel{value}
  \UnaryInfC{\((v, s, h, \sigma) \Downarrow (v, h, \sigma)\)}
  \DisplayProof{}
  \quad
  \AxiomC{}
  \LeftLabel{variable}
  \UnaryInfC{\((x, s, h, \sigma) \Downarrow (s[x], h, \sigma)\)}
  \DisplayProof{}
\end{center}

\begin{center}
  \AxiomC{\((N, s, h_0, \sigma_0) \Downarrow (v', h_1, \sigma_1)\)}
  \AxiomC{\((M, s, h_1, \sigma_1) \Downarrow (\lambda x. L, h_2, \sigma_2)\)}
  \AxiomC{\((L, s[x := v'], h_2, \sigma_2) \Downarrow (v, h_3,
    \sigma_3)\)}
  \LeftLabel{application}
  \TrinaryInfC{\((M\:N, s, h_0, \sigma_0) \Downarrow (v, h_3, \sigma_3)\)}
  \DisplayProof{}
\end{center}

\begin{prooftree}
  \AxiomC{\((e, s, h, \sigma) \Downarrow (v, h', \sigma')\)}
  \LeftLabel{unary operators}
  \RightLabel{\quad\(\circ \in \{\neg_\B, -_\N\}\)}
  \UnaryInfC{\((\circ e, s, h, \sigma) \Downarrow (\circ v, h',
    \sigma')\)}
\end{prooftree}

\begin{prooftree}
  \AxiomC{\((e, s, h_0, \sigma_0) \Downarrow (v, h_1, \sigma_1)\)}
  \AxiomC{\((e', s, h_1, \sigma_1) \Downarrow (v', h_2, \sigma_2)\)}
  \LeftLabel{binary operators}
  \BinaryInfC{\((e \circ e', s, h_0, \sigma_0) \Downarrow (v \circ v',
    h_2, \sigma_2)\)}
\end{prooftree}

Where
\(\circ \in \{+_\N, -_\N, *_\N, /_\N, <_\N, >_\N, \leq_\N, \geq_\N,
=_{\N, \B, \textbf{Strings}}, \neq_{\N, \B, \textbf{Strings}},
\vee_\B, \wedge_\B, concat_{\textbf{Strings}}\}\).

\begin{prooftree}
  \AxiomC{\((b, s, h_0, \sigma_0) \Downarrow (\top, h_1, \sigma_1)\)}
  \AxiomC{\((e, s, h_1, \sigma_1) \Downarrow (v, h_2, \sigma_2)\)}
  \LeftLabel{if-true-then}
  \BinaryInfC{\((\text{if } b \text{ then } e \text{ else } e', s, h,
    \sigma) \Downarrow (v, h_2, \sigma_2)\)}
\end{prooftree}

\begin{prooftree}
  \AxiomC{\((b, s, h_0, \sigma_0) \Downarrow (\bot, h_1, \sigma_1)\)}
  \AxiomC{\((e', s, h_1, \sigma_1) \Downarrow (v, h_2, \sigma_2)\)}
  \LeftLabel{if-false-else}
  \BinaryInfC{\((\text{if } b \text{ then } e \text{ else } e', s, h,
    \sigma) \Downarrow (v, h_2, \sigma_2)\)}
\end{prooftree}

\begin{center}
  \AxiomC{\(((\lambda x. e')\:e, s, h, \sigma) \Downarrow (v, h',
    \sigma')\)}
  \LeftLabel{let in}
  \UnaryInfC{\((\text{let } x = e \text{ in } e', s, h, \sigma)
    \Downarrow (v, h', \sigma')\)}
  \DisplayProof{}
  \quad
  \AxiomC{\((e, s, h, \sigma) \Downarrow (v, h', \sigma')\)}
  \AxiomC{print \(v\)}
  \LeftLabel{print}
  \BinaryInfC{\((\text{print } e, s, h, \sigma) \Downarrow (unit, h',
    \sigma')\)}
  \DisplayProof{}
\end{center}

\begin{center}
  \AxiomC{\((e, s, h, \sigma) \Downarrow (v, h', \sigma')\)}
  \AxiomC{\(\alpha \not\in h\)}
  \LeftLabel{ref}
  \BinaryInfC{\((\text{ref } e, s, h, \sigma) \Downarrow (unit,
    h'[\alpha := v], \sigma')\)}
  \DisplayProof{}
  \\\vspace{0.3cm}
  \AxiomC{\((e, s, h_0, \sigma_0) \Downarrow (\alpha, h_1, \sigma_1)\)}
  \AxiomC{\((h_1[\alpha], s, h_1, \sigma_1) \Downarrow (v, h_2,
    \sigma_2)\)}
  \LeftLabel{unref}
  \BinaryInfC{\((\text{unref } e, s, h_0, \sigma_0) \Downarrow (v,
    h_2, \sigma_2)\)}
  \DisplayProof{}
  \\\vspace{0.3cm}
  \AxiomC{\((e', s, h_0, \sigma_0) \Downarrow (v, h_1, \sigma_1)\)}
  \AxiomC{\((e, s, h_1, \sigma_1) \Downarrow (\alpha, h_2,
    \sigma_2)\)}
  \LeftLabel{assign}
  \BinaryInfC{\((e := e', s, h_0, \sigma_0) \Downarrow (unit,
    h_2[\alpha := v], \sigma_2)\)}
  \DisplayProof{}
\end{center}

\begin{prooftree}
  \AxiomC{\(\sigma' = (k\:id) :: t :: \sigma\)}
  \AxiomC{\(\tau = (t', s') \in \sigma'\)}
  \AxiomC{\((t', s', h, \sigma' \setminus \{\tau\}) \Downarrow (v,
    h'', \sigma'')\)}
  \LeftLabel{fork}
  \TrinaryInfC{\((k\:(\text{fork } t), s, h, \sigma) \Downarrow (v,
    h'', \sigma'')\)}
\end{prooftree}

\begin{prooftree}
  \AxiomC{\(\sigma' = (k\:id) :: \sigma\)}
  \AxiomC{\(\tau = (t', s') \in \sigma'\)}
  \AxiomC{\((t', s', h, \sigma' \setminus \{\tau\}) \Downarrow (v,
    h'', \sigma'')\)}
  \LeftLabel{yield}
  \TrinaryInfC{\((k\:\text{yield}, s, h, \sigma) \Downarrow (v,
    h'', \sigma'')\)}
\end{prooftree}

\begin{prooftree}
  \AxiomC{\((e, s, h, \sigma) \Downarrow (\top, h, \sigma)\)}
  \AxiomC{\((k\:id, s, h, \sigma) \Downarrow (v, h', \sigma')\)}
  \LeftLabel{wait-true}
  \BinaryInfC{\((k\:(\text{wait } e), s, h, \sigma) \Downarrow (v, h',
    \sigma')\)}
\end{prooftree}

\begin{prooftree}
  \AxiomC{\((e, s, h, \sigma) \Downarrow (\bot, h, \sigma)\)}
  \AxiomC{\(((k\:(\text{wait } e)) \text{ yield}, s, h, \sigma)
    \Downarrow (v, h', \sigma')\)}
  \LeftLabel{wait-false}
  \BinaryInfC{\((k\:(\text{wait } e), s, h, \sigma) \Downarrow (v, h',
    \sigma')\)}
\end{prooftree}
\end{document}
